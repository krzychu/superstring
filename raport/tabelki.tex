
#najlepszy
\begin{tabular}{c|ccc}
  \hline
  liczba słów & ewolucyjny & random & heurystyka\\
  \hline
  100 & 49985 & 80775 & 49318 \\
  200 & 50000 & 82697 & 49543 \\
  400 & 57209 & 83779 & 49823\\
  600 & 58640 & 84110 & 49764 \\
  800 & 63133 & 83178 & 49571 \\
  1000 & 65681 & 83297 & 49938 
\end{tabular}

#średni
\begin{tabular}{c|cc}
  \hline
  liczba słów & ewolucyjny & random\\
  \hline
  100 & 49985 & 80775 \\
  200 & 50194 & 82768 \\
  400 & 57893 & 83779 \\
  600 & 58732 & 84120 \\
  800 & 63152 & 83183 \\
  1000 & 65858 & 83301
\end{tabular}

#odchylenie
\begin{tabular}{c|cc}
  \hline
  liczba słów & ewolucyjny & random\\
  \hline
  100 & 0 & 0 \\
  200 & 258 & 40,7 \\
  400 & 530 & 0 \\
  600 & 176 & 15,8 \\
  800 & 42 & 10 \\
  1000 & 94 & 6,5
\end{tabular}


#Generator benchmarków

\subsection{Generowanie bechamarków}
Ze względu na trudność w znalezieniu benchmarków postanowiliśmy wygenerować je
na własną rękę.  Funkcja generująca przykładowy problem do rozwiązania pobiera
3 argumenty: oczekiwaną długość superstringu, liczbę słów które ma utworzyć
oraz procentowy overlap. Benchmark generowany jest w następujący sposób:

\begin{enumerate}
  \item utwórz superstring o zadanej długości
  \item podziel superstring na zadaną liczbę słów o tej samej długości
  \item każde ze słów rozszerz o podany współczynnik zmodyfikowany losowo
  w taki sposób, aby dalej pozostawało podsłowem
  \item losowo pozamieniaj kolejność otrzymanych słów
\end{enumarate}


Dla uproszczenia testowania algorytmów w wygenerowanych zestawach testowych
wybraliśmy długość stringu równą $50000$ oraz procentowy overlap równy $70\%$
dla każdego zestawu.  Mieliśmy nadzieję, że takie współczynniki wystarczą do
tego, by oczekiwana długość superstringu mogła w każdym zestawie testowym
posłużyć za wartość optymalną. Ale algorytmy potrafiły znaleźć jeszcze lepsze
wartości. Dzieje się tak ze względu na niewielki rozmiar alfabetu, jakiego
użyliśmy.  Jednakże znajdowane rozwiązania nigdy nie były znacznie lepsze od
oczekiwanego, a ponieważ nigdy nie możemy mieć pewności, że zadana długość
superstringu będzie optymalnym  rozwiązaniem, nawet dla bardzo dużego alfabetu
i overlapów, zdecydowaliśmy się zostawić dane benchmarki i posługiwać się
wartością $50000$ jako orientacyjną.


