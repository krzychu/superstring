\documentclass[10pt]{beamer}
\usetheme{Warsaw}

\usepackage[polish]{babel}
\usepackage[utf8]{inputenc}
\usepackage[T1]{fontenc}

\usepackage{parskip}
\usepackage{latexsym,gensymb,amsmath,amssymb,amsthm}
\usepackage{graphicx}
\usepackage{url}

\usepackage{graphics}
\usepackage{graphicx}
\usepackage{hyperref}

\title{Superstring}
\author{Krzysztof Chrobak\and Jan Sochiera}

\institute{Algorytmy Ewolucyjne 2012/2013}
\subject{Computational Sciences}


\begin{document}

\begin{frame}
\titlepage
\end{frame}

\begin{frame}{Algorytm}
\begin{itemize}
    \item <1-> Bazuje na naszym pierwszym projekcie
    \item <2-> SGA z przeszukiwaniem lokalnym i imigracją
    \item <3-> Ocena osobnika przy pomocy szybkiego algorymu opartego o KMP i drzewo sufiksowe
\end{itemize}
\end{frame}

\begin{frame}{Postęp prac}
Zaimplementowaliśmy:
\begin{itemize}
  \item <1-> Szybkie ocenianie osobników
  \item <2-> Generowanie losowych instancji problemu o zadanych parametrach
  \item <3-> Algorytm heurystyczny
\end{itemize}
\end{frame}

\begin{frame}{Przeprowadzone testy}
  \begin{center}
  \begin{tabular}{c|c|c|c}
    nr testu & ewolucyjny & losowy & heurystyka\\
    \hline
    10, 300 & 300 & 301 & 248\\
    20, 400 & 395 & 503 & 360\\
    30, 600 & 600 & 819 & 574\\
    40, 800 & 800 & 1052 & 766\\
    60, 3000 & 3200 & 4563 &  2916\\
  \end{tabular}
  \end{center}
\end{frame}

\begin{frame}{Co zostało do zrobienia}
\begin{itemize}
  \item <1-> Przyspieszenie oceniania osobników
  \item <2-> Wypróbowanie różnych metod skalowania funkcji przystosowania
\end{itemize}
\end{frame}

\end{document}








