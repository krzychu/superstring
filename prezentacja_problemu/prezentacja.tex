\documentclass[10pt]{beamer}
\usetheme{Warsaw}

\usepackage[polish]{babel}
\usepackage[utf8]{inputenc}
\usepackage[T1]{fontenc}

\usepackage{parskip}
\usepackage{latexsym,gensymb,amsmath,amssymb,amsthm}
\usepackage{graphicx}
\usepackage{url}

\usepackage{graphics}
\usepackage{graphicx}
\usepackage{hyperref}

\title{Superstring}
\author{Krzysztof Chrobak\and Jan Sochiera}

\institute{Algorytmy Ewolucyjne 2012/2013}
\subject{Computational Sciences}


\begin{document}

  \frame
  {
    \titlepage
  }

  \section{Problem superstringu}
  \subsection{Definicja problemu}
  \frame
  {
    \frametitle{Definicja problemu}

    \begin{itemize}
	    \item<1-> Dane : $s_1, s_2, s_3 \ldots s_n$ - słowa
	    \item<2-> Szukamy najkrótszego słowa, w którym zawierają się wszystkie $s_i$
    \end{itemize}
  }

  \subsection{Przestrzeń poszukiwań}
  \frame
	{
    \frametitle{Przestrzeń poszukiwań}
    Weźmy dowolny superstring $S$ dla $s_1, s_2, s_3 \ldots s_n$. Przez $b_1, \ldots b_n$ oznaczmy 
    pozycje, na których odpowidnie słowa zaczynają się w $S$. Przez sklejenie dwóch słów $a$ i $b$
    rozumiemy słowo $ab'$, gdzie $b'$ to sufiks $b$ pozostały po usunięciu z niego najdłuższego prefiksu
    będącego jednocześnie sufiksem $a$. Zauważmy, że gdybyśmy sklejali słowa zgodnie z niemalejącą 
    kolejnością $b_i$, dostalibyśmy właśnie $S$ (ignorując te, które już są podsłowami formowanego superstringu). 
    Kolejność wyznacza jednoznacznie otrzymany przez sklejanie
    superstring, zatem przestrzenią przeszukiwania jest $S_n$.
	}

  \subsection{Funkcja celu}
  \frame
	{
	  \frametitle{Funkcja celu}
    \begin{itemize}
      \item <1-> $f : S_n \rightarrow N$
      \item <2-> $f(\Pi) = $ długość superstringu uzyskanego poprzez sklejanie 
          $s_1, s_2, s_3 \ldots s_n$ w kolejności $\Pi$
    \end{itemize}
	}

  \section{Zastosowania w biologii}
  \frame
  {
    \frametitle{Problem superstringu w biologii}
    Znajdowanie superstringu jest wykorzystywane do sekwencjonowania DNA zgodnie z następującą procedurą (shotgun method):
    \begin{enumerate}
      \item <1-> Chemiczne rozbicie wielu egzemplarzy DNA na krótsze łańcuchy
      \item <2-> Odczytanie sekwencji zasad azotowych krótkich fragmentów
      \item <3-> Znalezienie najkrótszego słowa nad alfabetem $\Sigma = \{A, C, T, G\}$, 
          które zawiera odczytane fragmenty jako podsłowa
    \end{enumerate}
  }

\end{document}








